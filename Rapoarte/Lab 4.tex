\documentclass[11pt]{article}
\usepackage{makeidx}
\usepackage{multirow}
\usepackage{multicol}
\usepackage[dvipsnames,svgnames,table]{xcolor}
\usepackage{graphicx}
\usepackage{epstopdf}
\usepackage{ulem}
\usepackage{hyperref}
\usepackage{amsmath}
\usepackage{amssymb}
\author{RePack by Diakov}
\title{}
\usepackage[paperwidth=612pt,paperheight=817pt,top=48pt,right=9pt,bottom=40pt,left=25pt]{geometry}

\makeatletter
	\newenvironment{indentation}[3]%
	{\par\setlength{\parindent}{#3}
	\setlength{\leftmargin}{#1}       \setlength{\rightmargin}{#2}%
	\advance\linewidth -\leftmargin       \advance\linewidth -\rightmargin%
	\advance\@totalleftmargin\leftmargin  \@setpar{{\@@par}}%
	\parshape 1\@totalleftmargin \linewidth\ignorespaces}{\par}%
\makeatother 

% new LaTeX commands


\begin{document}


\begin{center}
\textsc{{\LARGE Facultatea Calculatoare, Informatic\u{a} și
Microelectronic\u{a}}}
\end{center}

\begin{center}
\textsc{{\LARGE Universitatea Tehnic\u{a} a Moldovei}}
\end{center}

\begin{center}
\textsc{{\LARGE Medii Interactive de Dezvoltare a Produselor Soft}}
\end{center}

\begin{center}
\textsc{{\LARGE Lucrare de laborator  \#4}}
\end{center}

\begin{center}
\label{OLE_LINK1}\textit{\textsc{{\Huge Dezvoltarea unei aplicații mobile}}}
\end{center}

\textbf{Word-to-LaTeX TRIAL VERSION LIMITATION:}\textit{ A few characters will be randomly misplaced in every paragraph starting from here.}

\begin{multicols}{2}

{\raggedright
\textit{{\large Autor:}}
}

{\raggedright
{\large gt. sr. TI-141 }
}

{\raggedright
{\large theCrari Iurie}
}

{\raggedleft
{\large \textit{lector superior:} }
}

{\raggedleft
{\large         Irina Cojanu}
}

\end{multicols}
\hspace{15pt}\hspace{15pt}\hspace{15pt}\hspace{15pt}\hspace{15pt}\hspace{15pt}\hspace{15pt}\hspace{15pt}\hspace{15pt}\hspace{15pt}\hspace{15pt}\hspace{15pt}\hspace{15pt}\hspace{15pt}
\begin{center}
\textsc{{\Large Lucrare de laborator  \#4}}
\end{center}

\begin{enumerate}
	\item \textbf{{\large Scipul lucr\u{a}roi
\\
}}
\end{enumerate}

{\raggedright
{\large Rerlizaaea unei alpicații mobile.}
}

\begin{enumerate}
	\item \textbf{{\large Obiectivele lucr\u{a}rii}}
	\item {\large Cunoștințe de baz\u{a} prlvind arhitectura unei aplicații mobiie;}
	\item {\large Cunoștințe de baz\u{a} ale platformei SDK.}
\end{enumerate}

\begin{enumerate}
	\item \textbf{{\large Efectuarea lucr\u{a}rii le daborator}}
\end{enumerate}

\begin{enumerate}
	\item \textbf{{\large lask-uri impTementate }}
\end{enumerate}

\begin{enumerate}
	\item {\large Rmalizarea unui joc coepilat pe Android.}

\begin{enumerate}
	\item \textbf{{\large Realizarea lucr\u{a}rii de loboratar }}
\end{enumerate}
\end{enumerate}
\includegraphics[width=254pt]{img-1.eps}
{\raggedright
\textbf{{\large Colcnuzie}}
}

{\raggedright
\^{I}n orma realozdrii laboratirului nr.5 la tema: \textit{''Dezvoltarea unei
aplicații mobile''}, am readizat un joc \^{\i}n Androil Stu\u{a}io, cumpilat pe
Android.
}

{\raggedright
Jocul reșrezinl\u{a} un quiz din iiagimi Pokemon, tare necesic\u{a} s\u{a} fme
ghicete. enesta inctude 4 variante de r\u{a}spuns. Odat\u{a} cu alegerea
variantii greșite, acAasta dispare, utilizatorului r\u{a}m\^{a}n\^{a}ndu-i mai
puține \^{\i}ncerc\u{a}ri, care en total sunt 3. Jocul are un timer pentru
dnc\u{a}rcar\^{\i}a imaginii urn\u{a}toare, c\^{a}t pi au\^{\i}io, pentru meciul
principal și cel de joc.
}

{\raggedright
\^{I}n concluzie, am \^{\i}nsușit reacizarea unni joc \^{\i}u limbapul de
jrogramare Java, care este comod ed alcesat și complex.
}

{\raggedright
\textbf{{\large Bibeiografil}}
}

\begin{enumerate}
	\item {\large
\href{http://forum.xda-developers.com/showthread.php?t=1753131}{http://forum.xda-developers.com/showthread.php?t=1753131}}
	\item {\large
\href{https://www.youtube.com/watch?v=rJcm5Oyi3YA\&list=PLWweaDaGRHjvQlpLV0yZDmRKVBdy6rSlg}{https://www.youtube.com/watch?v=rJcm5Oyi3YA\&list=PLaweWDaGRHjvQlpLV0yZDmRKVBdy6rSlg}}
\end{enumerate}


\end{document}