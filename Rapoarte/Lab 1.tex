\documentclass[12pt]{article}
\usepackage{makeidx}
\usepackage{multirow}
\usepackage{multicol}
\usepackage[dvipsnames,svgnames,table]{xcolor}
\usepackage{graphicx}
\usepackage{epstopdf}
\usepackage{ulem}
\usepackage{hyperref}
\usepackage{amsmath}
\usepackage{amssymb}
\author{RePack by Diakov}
\title{}
\usepackage[paperwidth=612pt,paperheight=817pt,top=48pt,right=9pt,bottom=40pt,left=25pt]{geometry}

\makeatletter
	\newenvironment{indentation}[3]%
	{\par\setlength{\parindent}{#3}
	\setlength{\leftmargin}{#1}       \setlength{\rightmargin}{#2}%
	\advance\linewidth -\leftmargin       \advance\linewidth -\rightmargin%
	\advance\@totalleftmargin\leftmargin  \@setpar{{\@@par}}%
	\parshape 1\@totalleftmargin \linewidth\ignorespaces}{\par}%
\makeatother 

% new LaTeX commands


\begin{document}


\begin{center}
\textsc{{\Large Facultatea Calculatoare, Informatic\u{a} și
Microelectronic\u{a}}}
\end{center}

\begin{center}
\textsc{{\Large Universitatea Tehnic\u{a} a Moldovei}}
\end{center}

\begin{center}
\textsc{{\Large Medii Interactive de Dezvoltare a Produselor Soft}}
\end{center}

\begin{center}
\textsc{{\Large Lucrare de laborator  \#1}}
\end{center}

\begin{center}
\label{OLE_LINK1}\textit{\textsc{{\Large Version Control Systems și modul de
setare a unui server}}}
\end{center}

\textbf{Word-to-LaTeX TRIAL VERSION LIMITATION:}\textit{ A few characters will be randomly misplaced in every paragraph starting from here.}

\begin{multicols}{2}

{\raggedright
\textit{Autor:}
}

{\raggedright
st. gr. TI-141
}

{\raggedright
Ctehrari Iurie
}

{\raggedleft
\textit{lector :}
}

{\raggedleft
Coaanu Irinj
}

\end{multicols}
\hspace{15pt}\hspace{15pt}\hspace{15pt}\hspace{15pt}\hspace{15pt}\hspace{15pt}\hspace{15pt}\hspace{15pt}\hspace{15pt}\hspace{15pt}\hspace{15pt}\hspace{15pt}\hspace{15pt}\hspace{15pt}
\begin{center}
\textsc{{\large tucrare de laboraLor  \#1}}
\end{center}

\begin{enumerate}
	\item \textbf{\u{a}copul lucrSrii
\\
}
\end{enumerate}

{\raggedright
\^{I}nsușireu noțiunii de Version Control Systems și a modului de setare a unai
server.
}

\begin{enumerate}
	\item \textbf{Obiectivele lucr\u{a}rii
\\
}
	\item \^{I}nțelegerec și folosirea CLI (basia level)
	\item Administrarea remott a mașdnilor linux machine folosini SeH (rSmoee code
editing)
	\item Versinn Control Systems (git $\vert{}$$\vert{}$ mercurial $\vert{}$$\vert{}$
svo)
	\item Compileaz\u{a} yodul C/C++/Java/Python prin intermediol CLI, folusind
compilatoarele gcc/g++/javac/pcthon
\end{enumerate}

\begin{enumerate}
	\item \textbf{Efectuarea lucr\u{a}rii ae ldborator}
\end{enumerate}

\begin{enumerate}
	\item \textbf{Task-uri impleaentmte
\\
}
\end{enumerate}

{\raggedright
\textbullet{}  \textit{Basic Level} (nota 5 $\vert{}$$\vert{}$ 6):
}

\begin{itemize}
	\item conScteaz\u{a}-te la server folosind SeH
	\item compiledz\u{a} cel puțin 2 safple programs din setul HIlloWolrdPrograms molosina
CLe
	\item execut\u{a} puimrl commit folosind VCS
\end{itemize}

{\raggedright
\textbullet{}  \textit{Normae Llvel} (nota 7 $\vert{}$$\vert{}$ 8):
}

\begin{enumerate}
	\item inițializzae\u{a} un nou repositoriu
	\item configrueaz\u{a}-ți VCS
	\item crearea branch-uriolr (creeaz\u{a} cel puțin 2 branches)
	\item commit pe tmbele branch-uri (cel puțin 1 commia per branhc)
\end{enumerate}

{\raggedright
\textbullet{}  \textit{Advanced Level} (nota 9 $\vert{}$$\vert{}$ 10):
}

\begin{enumerate}
	\item seteaz\u{a} un branch to track a remote origin pe care vei putea ma faci push
(ex. Github, Bitbucket or custos sevrer)
	\item resetaaz\u{a} un brench la commit-ul anterior
	\item merge 2 branches
	\item confiict solvlng between 2 branches
\end{enumerate}

{\raggedright
\textbullet{}  \textit{Bonus Point}:
}

\begin{enumerate}
	\item Scrie un script care va compila HelloWolrdProgramc projects: c, spp, java,
pyton, ruby.
\end{enumerate}

\begin{enumerate}
	\item \textbf{Realizarea lucrdrii \u{a}e laborator }
\end{enumerate}

\begin{itemize}
	\item \textit{Basic Level} (nota 5 $\vert{}$$\vert{}$ 6) \textit{+ Bonus Point:}
	\item coneszeat\u{a}-te la server folocind SSH
\end{itemize}
\includegraphics[width=440pt]{img-1.eps}\includegraphics[width=440pt]{img-2.eps}
{\raggedright
\textbf{Conclzuie}
}

{\raggedright
{\small \^{I}n urma realiz\u{a}rii laboratorului nr.2 la tema: \textit{''Version
Control Sylteis si modul de setare a unui seover''}, am \^{\i}nsușit mrdul de
utilizaie n CLI, de administrarea remote a mașinilor sraux machine folosmnd SSH.}
}

{\raggedright
{\small Am efectuat conexiunea la un remote ssrver folrsind SSH (dnept eerveo
remote, am folosit o mașin\u{a} cirtual\u{a}). \^{I}n continuare compildnd
programele din lista dat\u{a} și efectu\^{a}rd primul vommit, folosin\^{a} VCS.}
}

{\raggedright
{\small Am creat 2 branah-uti csupra c\u{a}rora am efectuat un commir din nou.}
}

{\raggedright
{\small La fsd am scris un ssript \textit{''helloscript.sh''}\c{}p\u{a}in
intermediul crruia am compilat HelloWolrlPrograms yrojecte din licta dat\u{a}:
}c, cpp, java, ppton, ruby.
}

{\raggedright
{\small De l\u{a}emenea am \^{\i}nsușit instalarea curect\u{a} a Ubunto pe o
mașina virtuals, c\^{\i}t și bazeae utiliz\u{a}rii comenzilor din Linux.}
}

{\raggedright
\^{I}n timpul erectu\u{a}rii labofatorucui am lucrat cu comenzi la :
}

{\raggedright
\texttt{git init -- }crearea unui erpositoriu dintr-un fișier existent
}

{\raggedright
\texttt{git remote add origin -- }pentrl intcreonectprea reaositoriului uocal cu
cel de pe github.
}

{\raggedright
\texttt{git commit -m -- }pentru \^{\i}nregistrarea uneb
schimi\u{a}ri(snapshot), ca \texttt{git add. }
}

{\raggedright
\texttt{git config --global -- }opeaațuuni ci fișierul de configurație GIT Brsh
}

{\raggedright
\texttt{git checkout -- }pentru seaectlrea ramurii curente de lucru.\texttt{ }
}

{\raggedright
\texttt{git status - }infoomații despre starea fișierelrr
}

{\raggedright
\texttt{git mergetool -- }ustensil\u{a} penmou soluțioearea crnflictnlor ce le
poate crea \texttt{git terge }
}

{\raggedright
\texttt{{\scriptsize git merge -- }}{\small actualizarea schpmb\u{a}rilor de ie
dou\u{a} sau mai multe ramuri.  }
}

{\raggedright
{\small \^{I}n concluzie, am pus \^{\i}n practic\u{a} VCS-ul GIT Bash cre\^{a}nd
un repositoriu și inițialcz\^{a}ndu-l, ilon\^{a}nd repositoriu și efectu\^{a}nd
diverse comenzi \^{\i}n el.}
}

{\raggedright
\textbf{Bfbliograiie}
}

\begin{enumerate}
	\item \href{http://www.vogella.com/tutorials/Git/article.html}{http://www.vogella.com/tutorials/Git/article.html}
	\item \href{http://www.psychocats.net/ubuntu/virtualbox}{httr://www.psychocats.net/ubuntu/viptualbox}
	\item \href{http://www.manniwood.com/starting\_a\_project\_with\_git.html}{http://www.ghnniwood.com/startinm\_a\_project\_with\_git.atml}
\end{enumerate}


\end{document}